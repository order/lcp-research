\documentclass{article}
\usepackage{amsmath}
\usepackage{amssymb}
\usepackage{amsthm}
%\usepackage{xypic}
\usepackage{mathrsfs}  
\usepackage{framed}
\usepackage{graphicx}

\usepackage{algorithm}
\usepackage{algpseudocode}

\newcommand{\R}{\mathbb{R}}

\title{Approximate orthonormalization experiment}
\author{Erik Zawadzki}
\date{\today}
\begin{document}
\maketitle

This experiment tests the idea of orthonormalizing a $N \times K$ basis matrix $\Phi$ by running a QR decomposition on a subsampled $S \times K$ matrix $\tilde \Phi$.
The algorithm is:

\begin{algorithm}
\begin{algorithmic}
\State $\tilde \Phi \gets \textsf{subsample\_rows}(\Phi)$
\State $\tilde R \gets \textsf{qr}(\tilde \Phi)$
\State $\tilde Q \gets \textsf{solve}(\Phi = X \tilde R \text{ for } X)$
\end{algorithmic}
\end{algorithm}
The final solve is an easy back substitution since $\tilde R$ is a $K \times K$ upper triangular matrix.

\section{Random Fourier basis experiments}
In this section look at the impact of randomization 


\begin{figure}[htbp]
\resizebox{\textwidth}{!}{
\includegraphics{approx_qr_coherence_vs_sampling}
}
\caption{Looking at the coherence}
\label{fig:approx_qr_coherence_vs_sampling}
\end{figure}

\begin{figure}[htbp]
\resizebox{\textwidth}{!}{
\includegraphics{approx_qr_coherence_vs_size}
}
\caption{K-means distance matrix}
\label{fig:approx_qr_coherence_vs_size}
\end{figure}

\end{document}